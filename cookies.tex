\documentclass[a4paper]{article}
\usepackage{hyperref}
%opening
\title{On Cookies}
\author{Members of RTG 2450}

\begin{document}

\maketitle

\begin{abstract}
	Cookies are a ubiquitous and delicious form of baked good. 
	In this paper, we summarize some important facts about them.
\end{abstract}

\section{Introduction}
A cookie is a baked or cooked food that is typically small, flat and sweet. 
It usually contains flour, sugar and some type of oil or fat. 
It may include other ingredients such as raisins, oats, chocolate chips, nuts, etc. 
Cookies are most commonly baked until crisp or just long enough that they remain soft, but some kinds of cookies are not baked at all.

\subsection{Notable Cookie Varieties from Around the World}
\begin{itemize}
	\item Butter cookie
	\item Chocolate cookie
	\item Vanillekipferl (German vanilla cookies)
	\item Springerle
	\item Florentine biscuit
	\item Rock cake
	\item \dots
\end{itemize}

\section{Terminology and Etymology}
In most English-speaking countries outside North America, including the United Kingdom, the most common word for a crisp cookie is biscuit. 
The term cookie is normally used to describe chewier ones. 
However, in many regions both terms are used. 
In Scotland the term cookie is sometimes used to describe a plain bun. 
The German term \textit{Keks} derives from the English word cakes.

Its American name derives from the Dutch word \textit{koekje} or more precisely its informal, dialect variant \textit{koekie} which means little cake, and arrived in American English with the Dutch settlement of New Netherland, in the early 1600s.


\section{Traditional uses for cookies}
In many countries, cookies are dunked into hot beverages.
Often, specific cookies are made for specific occasions like Christmas or Valentine's day

\section{General Theory of Cookies}
Despite its descent from cakes and other sweetened breads, the cookie in almost all its forms has abandoned water as a medium for cohesion. 
Water in cakes serves to make the base (in the case of cakes called "batter") as thin as possible, which allows the bubbles – responsible for a cake's fluffiness – to better form. 
In the cookie, the agent of cohesion has become some form of oil. 
Oils, whether they be in the form of butter, vegetable oils, or lard, are much more viscous than water and evaporate freely at a much higher temperature than water. 
Thus a cake made with butter or eggs instead of water is far denser after removal from the oven.

Oils in baked cakes do not behave as soda tends to in the finished result. 
Rather than evaporating and thickening the mixture, they remain, saturating the bubbles of escaped gases from what little water there might have been in the eggs, if added, and the carbon dioxide released by heating the baking powder. 
This saturation produces the most texturally attractive feature of the cookie, and indeed all fried foods: crispness saturated with a moisture (namely oil) that does not sink into it. 

\subsection{Sample Cookie Recipe}
There are too many to choose from! Fill in your favorite!

\section{Conclusion and Outlook}
As you see, cookies are awesome. Go bake some!

\section{References}
Wikipedia article on cookies: \url{https://en.wikipedia.org/wiki/Cookie}

\end{document}
